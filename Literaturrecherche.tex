\documentclass[a4paper,12pt]{article}
\usepackage[utf8]{inputenc}
\usepackage[ngerman]{babel}
\usepackage[T1]{fontenc}
\usepackage{amsmath}

\begin{document}
1.Literaturrecherche\\
Wie funktioniert die Gewinnung elektrischer Energie mit Hilfe von Photovoltaik?\\
Um eine Gewinnung elektrischer Energie mithilfe von Photovoltaik zu ermöglichen, braucht es unter anderem eine Solarzelle. Betrachten wir zum einfacheren Verständnis eine Photodiode, die in ihrer Wirkungsweise einer Solarzelle gleichkommt. 
Die Grundlage für die Photovoltaiktechnik liefert uns das zweite Bohrsche Postulat: "Der Übergang eines Elektrons von einer Schale zur anderen erfolgt unter Emission oder Absorption von elektromagnetischer Strahlung." Sprich: Trifft Licht auf ein Elektron, so wird dieses auf ein anderes Energieniveau angehoben. 

Stellen wir uns die Photodiode so vor, als bestünde sie aus zwei aneinandergesetzten Kristallen. Der eine n-dotiert (Elektronenüberschuss) und der andere p-dotiert (Elektronenmangel). Fügt man diese nun zusammen, so entsteht ein Gleichgewicht am Übergang, denn die überschüssigen Elektronen werden von den "Elektronenlöchern" im p-dotierten Teil angezogen. Zurück bleibt eine ortsfeste positive Ladung im n-dotierten Raum und eine ortsfeste negative Ladung im p-dotierten Raum. Diesen Ausgleich bezeichnet man elektrisches Feld. Durch die steigende Anzahl an ortsfesten Ladungen entsteht der sog. Diffusionsstrom. Dieser Strom und das el. Feld heben sich gegenseitig auf.
Durch das einfallende Licht werden nun die Elektronen im el. Feld zur n-dotierten Seite befördert. Bringt man jeweils einen Kontakt an die n-und p-dotierte Seite, lässt sich der sog. Photostrom abgreifen. 
Was versteht man unter dem Maximum-Power-Point (MPP) einer Solarzelle?\\
Eine Solarzelle kann je nach Betriebsbereich unterschiedliche Leistung abgeben. Der Punkt an dem die maximale Leistung erbracht wird, nennt man MAXIMUM-POWER-POINT (MPP). Da die Leistung das Produkt aus der Spannung U und dem Strom I ist, lässt sich aus der grafischen Darstellung der MMP als die größtmögliche Fläche unter dem Grafen der Leistung darstellen.
Welche Möglichkeiten zur Speicherung von (regenerativ erzeugter) elektrischer Energie
gibt es?\\
Hochenergiespeicher (Druckluftspeicher, Energiespeicherkraftwerke), Hochleistungsspeicher (Elektrochemische Doppelschichtkondensatoren, Schwungräder, Supraleitende Spulen), Elektrochemische Speichersysteme (Batterien, Wasserstoff...), Thermische Speicher\\

Solare Wechselrichter werden zwischen Solaranlagen und das Stromnetz geschaltet.
Warum bestimmen Solare Wechselrichter den MPP und wie häufig sollte das ein Wechselrichter
tun? Beantworten Sie diese Frage bitte in der Dokumentation und gehen Sie
dabei auf den Grund des Verfahrens ein und nicht auf die praktische Realisierung desselben.\\
 




\end{document}